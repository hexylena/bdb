\PassOptionsToPackage{unicode=true}{hyperref} % options for packages loaded elsewhere
\PassOptionsToPackage{hyphens}{url}
\PassOptionsToPackage{dvipsnames,svgnames*,x11names*}{xcolor}
%
\documentclass[paper=a4,justified,a4paper]{tufte-handout}
\usepackage{lmodern}
\usepackage{amssymb,amsmath}
\usepackage{cleveref}
\usepackage{ifxetex,ifluatex}
\usepackage{fixltx2e} % provides \textsubscript
\ifnum 0\ifxetex 1\fi\ifluatex 1\fi=0 % if pdftex
  \usepackage[T1]{fontenc}
  \usepackage[utf8]{inputenc}
  \usepackage{textcomp} % provides euro and other symbols
\else % if luatex or xelatex
  \usepackage{unicode-math}
  \defaultfontfeatures{Ligatures=TeX,Scale=MatchLowercase}
\fi
% use upquote if available, for straight quotes in verbatim environments
\IfFileExists{upquote.sty}{\usepackage{upquote}}{}
% use microtype if available
\IfFileExists{microtype.sty}{%
\usepackage[]{microtype}
\UseMicrotypeSet[protrusion]{basicmath} % disable protrusion for tt fonts
}{}
\IfFileExists{parskip.sty}{%
\usepackage{parskip}
}{% else
\setlength{\parindent}{0pt}
\setlength{\parskip}{6pt plus 2pt minus 1pt}
}
\usepackage{xcolor}
\usepackage{hyperref}
\hypersetup{
            pdftitle={Assignment 11 - Final Reflection Report},
            pdfauthor={Helena Rasche},
            colorlinks=true,
            linkcolor=Maroon,
            filecolor=Maroon,
            citecolor=Blue,
            urlcolor=Blue,
            breaklinks=true}
\urlstyle{same}  % don't use monospace font for urls
\usepackage{longtable,booktabs}
% Fix footnotes in tables (requires footnote package)
\IfFileExists{footnote.sty}{\usepackage{footnote}\makesavenoteenv{longtable}}{}
\setlength{\emergencystretch}{3em}  % prevent overfull lines
\providecommand{\tightlist}{%
  \setlength{\itemsep}{0pt}\setlength{\parskip}{0pt}}
\setcounter{secnumdepth}{0}
% Redefines (sub)paragraphs to behave more like sections
\ifx\paragraph\undefined\else
\let\oldparagraph\paragraph
\renewcommand{\paragraph}[1]{\oldparagraph{#1}\mbox{}}
\fi
\ifx\subparagraph\undefined\else
\let\oldsubparagraph\subparagraph
\renewcommand{\subparagraph}[1]{\oldsubparagraph{#1}\mbox{}}
\fi

% set default figure placement to htbp
\makeatletter
\def\fps@figure{htbp}
\makeatother

\usepackage{pdfpages}

%%%%%%%%%%% Header and Footer %%%%%%%%%%%%%%%%%%
\fancyfoot[CE,CO]{\flushright \includegraphics[width=3cm]{../avans.jpg}}
\fancyhead[CE,CO]{\flushleft \smallcaps{\today}}


\title[A11 - Final Reflection]{Assignment 11 - Final Reflection Report}
\author{Helena Rasche}
\date{2022-02-07}

\begin{document}
\maketitle
\noindent\rule{5in}{0.4pt}


\hypertarget{reflections}{%
\section{Reflections}\label{reflections}}

\hypertarget{learning-objective-1-lecturingteaching}{%
\subsection{Learning Objective 1:
Lecturing/teaching}\label{learning-objective-1-lecturingteaching}}

\begin{quote}
The participant performs a lesson on the basis of a lesson preparation
form, in such a way that he (sic) addresses strengths and executes
actions for improvement.
\end{quote}

With regards to LO1, I had never prepared a lesson preparation form
before this class, it was not a skill I'd exercised, nor had I ever seen
a lesson preparation form in real life! Over the course of this module I
think my skills in that activity have significantly improved, now I can
construct a useful lesson preparation form which can be re-used by other
teachers which accurately tracks what learning objectives students are
working towards in each portion of the class and which of the Revised
Bloom's taxonomy areas that task is exercising. I feel much more
confident in this task now and I think the product I've produced has
improved at each stage of this module as I had to produce more and more
lesson preparation forms.

Here is a comparison of lesson objectives I wrote at the start, and end,
of this module

\begin{longtable}[]{@{}ll@{}}
\toprule
\begin{minipage}[b]{0.47\columnwidth}\raggedright
Before\strut
\end{minipage} & \begin{minipage}[b]{0.47\columnwidth}\raggedright
After\strut
\end{minipage}\tabularnewline
\midrule
\endhead
\begin{minipage}[t]{0.47\columnwidth}\raggedright
Create a ``course'' object in CoCalc to learn this alternative system
for classroom management\strut
\end{minipage} & \begin{minipage}[t]{0.47\columnwidth}\raggedright
LO1: Compute multiple whole genome assemblies in such a way to develop
big data processing skills (Apply+Procedural)\strut
\end{minipage}\tabularnewline
\begin{minipage}[t]{0.47\columnwidth}\raggedright
Create an assignment to further developer R skills of previous
lessons\strut
\end{minipage} & \begin{minipage}[t]{0.47\columnwidth}\raggedright
LO2: Learn to evaluate quality metrics so that they can separate good
and bad assemblies (Analyse+Conceptual, Evaluate+Procedural)\strut
\end{minipage}\tabularnewline
\begin{minipage}[t]{0.47\columnwidth}\raggedright
Auto-grade the assignment to introduce a new feature of this digital
classroom which allows for more automation and saves them time.\strut
\end{minipage} & \begin{minipage}[t]{0.47\columnwidth}\raggedright
LO3: Visualise assemblies so that they understand presentation of
various failure modes (Apply+Procedural, Evaluate+Conceptual)\strut
\end{minipage}\tabularnewline
\bottomrule
\end{longtable}

Likewise, my lesson plans improved significantly as well:

\begin{longtable}[]{@{}lll@{}}
\toprule
\begin{minipage}[b]{0.21\columnwidth}\raggedright
Section\strut
\end{minipage} & \begin{minipage}[b]{0.28\columnwidth}\raggedright
Before\strut
\end{minipage} & \begin{minipage}[b]{0.42\columnwidth}\raggedright
After\strut
\end{minipage}\tabularnewline
\midrule
\endhead
\begin{minipage}[t]{0.21\columnwidth}\raggedright
Schedule\strut
\end{minipage} & \begin{minipage}[t]{0.28\columnwidth}\raggedright
30 min\strut
\end{minipage} & \begin{minipage}[t]{0.42\columnwidth}\raggedright
``Cold Open'' 20 minutes\strut
\end{minipage}\tabularnewline
\begin{minipage}[t]{0.21\columnwidth}\raggedright
Content\strut
\end{minipage} & \begin{minipage}[t]{0.28\columnwidth}\raggedright
What is CoCalc recap / What can it do\strut
\end{minipage} & \begin{minipage}[t]{0.42\columnwidth}\raggedright
We'll begin with the assembly exercise, students given either paper or
digital pieces of paper that they need to re-assemble into their
original sentence. Some will have mistakes or low coverage portions so
they can make some guesses\strut
\end{minipage}\tabularnewline
\begin{minipage}[t]{0.21\columnwidth}\raggedright
Teacher\strut
\end{minipage} & \begin{minipage}[t]{0.28\columnwidth}\raggedright
Presentation on CoCalc\strut
\end{minipage} & \begin{minipage}[t]{0.42\columnwidth}\raggedright
Activity introduction, Observation\strut
\end{minipage}\tabularnewline
\begin{minipage}[t]{0.21\columnwidth}\raggedright
Student\strut
\end{minipage} & \begin{minipage}[t]{0.28\columnwidth}\raggedright
Introductions, Listening, Questioning\strut
\end{minipage} & \begin{minipage}[t]{0.42\columnwidth}\raggedright
Students will apply existing knowledge from e.g.~legpuzzel solving to
assemble the sentences\strut
\end{minipage}\tabularnewline
\begin{minipage}[t]{0.21\columnwidth}\raggedright
Justify\strut
\end{minipage} & \begin{minipage}[t]{0.28\columnwidth}\raggedright
Students need to reactivate their knowledge from the previous lesson of
how R works.\strut
\end{minipage} & \begin{minipage}[t]{0.42\columnwidth}\raggedright
This goes towards \textbf{LO1}, learning about the procedure of whole
genome assembly by giving them a fun introductory activity where they
can transition from excitement of being in the class to a critical
thinking state and begin to \emph{Apply+Procedural} an algorithm and
begin to \emph{Analyze+Understand} the algorithm they're making
intuitively.\strut
\end{minipage}\tabularnewline
\bottomrule
\end{longtable}

I think from this it's clear my lesson plans have improved significantly
from their first iteration, based on the feedback of everyone involved.

\hypertarget{learning-objective-2-supporting-students}{%
\subsection{Learning Objective 2: Supporting
students}\label{learning-objective-2-supporting-students}}

\begin{quote}
The participant supports students in educational activities in such a
way that it encourages them to learn activities.
\end{quote}

I worry here I have only become marginally better, as I have not found
concrete ways to shift student motivations past ``well it affects my
grade''. As excited as I am about the power of the skills I teach in my
classes, the students lack real world problem applications and as such
cannot feel the same motivation.

I have learned to develop more engaging lessons, especially through the
tip of varying work styles every 7-15 minutes as student attention
wanes, using more modes of engaging the students. However this again has
been constrained by real world practicalities--some work modes are
simply inappropriate for the lesson--it's hard to have students write an
essay on how they feel about a programming language, it does not further
their understanding. So it's a new challenge for me to find engaging and
varied work forms that are appropriate to the lesson material, and fit
within the time constraints.

\hypertarget{learning-objective-3-designing-education}{%
\subsection{Learning Objective 3: Designing
education}\label{learning-objective-3-designing-education}}

\begin{quote}
The participant (re)designs an educational activity, in such a way that
it can be performed
\end{quote}

I think this was my favourite part of the class by a wide margin; the
lessons I teach need a lot of improvement and this class provided
dedicated time to work on that \emph{and} research the optimal way to do
that. I learned a lot about computer science teaching methodology and
the existing body of research investigating this topic in detail. I got
to practice those techniques in my classes, to great effect and very
positive student feedback. It gave me time and space to identify weak
points in the lessons and consider how I would fix those, I only wish I
had time to implement the changes, something that could have been done
in a similar amount of time to the required report, and would have
resulted in a significant improvement for students.

\hypertarget{learning-objective-4-professional-lectureship}{%
\subsection{Learning Objective 4: Professional
lectureship}\label{learning-objective-4-professional-lectureship}}

\begin{quote}
The participant reflects on his teaching position in such a way he (sic)
identifies his (sic) strengths and development points
\end{quote}

Since the start of the module I've become a lot stronger in rigorous
lesson design, applying Bloom's taxonomy to my learning objectives, and
breaking up lessons into smaller, more bite-sized chunks for students,
to keep their energy levels up. Unfortunately, as I expressed on the
first day, I'm weak in overall course design--connecting these lessons
together to form a greater whole, and on that topic I haven't progressed
due to the nature of this course. That said, this course has given me
ideas for more interactive activities I can do with my students and I
suspect they'll benefit from that.

\hypertarget{future-improvements}{%
\section{Future Improvements}\label{future-improvements}}

In the near future I will be working to

\begin{enumerate}
\def\labelenumi{\arabic{enumi}.}
\tightlist
\item
  Develop more interactive activities for students. These should help
  students engage better with the material, for students with different
  learning preferences or styles.
\item
  Redesign more lessons, in order to break them up into smaller chunks,
  and keep student attention high throughout the lesson.
\item
  Produce more lesson preparation forms to go along with our existing
  lessons, where they're currently missing.
\end{enumerate}

For item 1 above, I'm specifically excited about two activities I have
in mind. First an interactive assembly activity where students replicate
the assembly process done by computers in genomics, but by hand and on
paper. Here they get to work together with friends to solve a puzzle,
exactly like the computer does. I expect it will make that lesson
significantly more engaging and exciting for them to compete with each
other. And for the second, I plan to make their formative assessments a
bit more (anonymously) competitive so that they can see how their
solution performs (in terms of memory and computation time) compared to
other students. I hope this will give the competitive students
motivation to improve their solutions and find better options.

\hypertarget{feedback-received}{%
\section{Feedback Received}\label{feedback-received}}

I received a lot of good feedback throughout this module which I will
summarise here

\begin{itemize}
\tightlist
\item
  Apologise less, once is enough.
\item
  Use 3C learning objectives as LO context is important.
\item
  Include Do's and Don't's when discussing new techniques, to help other
  teachers and yourself later.
\item
  Include more graphics to help convey meaning, not everyone can read
  the text and find it sufficient.
\item
  Consider the audience, some terms need elaboration and explanation if
  they're consumed by a broader audience (e.g.~BDB reviewers.)
\item
  ``Can you help your friend'', get students to help each other if one
  is stuck.
\end{itemize}

This feedback came over the course of both lessons and peer-feedback
reports and I've found very useful as each assignment passed. Early
feedback from Widya had a significant improvement on my lesson
preparation forms, while Titia's comments on teaching reminder me to be
more confident in myself and my process. Reamflar's reports gave me
inspiration for improvements and a reminder that I was writing in an
overly insular and inaccessible way, something I do not want! Lastly
observing Franca's lesson was instrumental in improving my own and
learning new ways to engage with students.

\end{document}
